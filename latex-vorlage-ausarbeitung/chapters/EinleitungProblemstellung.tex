\chapter{Einleitung und Problemstellung}


Hier werden folgende Aspekte berücksichtigt: 


\begin{itemize}
	\item Relevanz für ein E-Commerce Shop. 
	\item Aktueller Stand im Markt: Welche ähnlichen Anwendungen gibt es und wie unterscheidet sich unsere Anwendung von den bisherigen? 
	\item Problem-/Fragestellung und Zielsetzung für ein zu entwickelndes  
\end{itemize}

\section{Überblick}\index{Überblick}

ShopNook ist ein internetbasierter Marktplatz, der darauf abzielt, Käufern einen großartigen und angenehmen Einkaufsprozess zu bieten. Mit dieser Anwendung können Käufer ganz einfach durch die verschiedenen verfügbaren Produkte gehen, mehrere davon in ihren Einkaufskörben sammeln und sie sicher bezahlen. Das Design von ShopNook ist attraktiv und auf jeder Bildschirmgröße zugänglich, so dass jeder Nutzer, der die Website besucht, unabhängig von seinen Vorlieben bei den Geräten angesprochen wird. Die Website verfügt außerdem über eingebaute Funktionen wie ein Bestandskontrollsystem, Admin-Rollen wie z. B. einen Editor für Produktdetails und andere sowie Käuferprivilegien.

\section{Globale Anforderungen}\index{Globale Anforderungen}

\subsection{Funktionale Anforderungen}

Die App wird über alle notwendigen Funktionen verfügen, um ein reibungsloses Einkaufserlebnis zu gewährleisten. Die Kunden werden die Möglichkeit haben, sich sicher zu registrieren, einzuloggen und einen Produktkatalog zu erkunden. Zu jedem Produkt werden vollständige Details wie Name, Beschreibung, Preis und Bilder angezeigt. Die Kunden können schnell Artikel in ihren Einkaufskorb legen, den Inhalt ändern und zur Zahlung übergehen. Sichere Kreditkartenzahlungen werden während des Kaufvorgangs über die Stripe-API abgewickelt. Administratoren können die Waren überwachen und die Bestellungen im Auge behalten, um sicherzustellen, dass das Online-Geschäft gut funktioniert. 

\subsection{Nicht-funktionale Anforderungen}

Die App wird effizient mit gleichzeitigen Nutzern umgehen und die Ladezeiten der Seite sind minimal. HTTPS-Verschlüsselung und JWT-basierte Authentifizierung stellen die Sicherheit in den Vordergrund. Die Einheitlichkeit der Geräte wird durch eine reaktionsschnelle und einfach zu bedienende Schnittstelle gewährleistet.

\subsection{Technische  Anforderungen}

Die vollwertige E-Commerce-App wird mit npm für JavaScript-Dependenzen und Maven für die Java Dependenzen und Build-Prozesse entwickelt. Im Frontend wird Angular verwendet, um eine dynamische und reaktionsschnelle Schnittstelle für die Benutzer bereitzustellen, während im Backend das Spring Boot-Framework verwendet wird, um RESTful APIs zu erstellen. MySQL wird die Datenbank sein, die Benutzer-, Produkt- und Bestelldaten speichert. Git und GitHub werden für die Versionskontrolle verwendet, um Änderungen zu verwalten und die Teamarbeit zu erleichtern. 