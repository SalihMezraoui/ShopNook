\chapter{Methodologie und Systementwurf }

Hier werden über folgende Punkte diskutiert:

\begin{itemize}
	\item Die Entwicklungskonzeption, die wir während der Implementation der Anwendung benutzt haben, sowie die Tools, Technologien und Methodologien. 
	\item Überblick über Spring Boot und seine wichtigste  Funktionalitäten, die auch in dem Projekt angewendet sein werden. 
	\item Detaillierte Erklärung der Struktur und Design der Anwendung (Bsp: Anlyse-/Entwurfsdiagram) 
	\item Datenbanken, UI-Design und unterschiedliche Komponenten des Systems. 
	\item Wie Spring Boot die Entwicklung und Implementation der verschiedenen Module vereinfacht. 
\end{itemize}


\section{Entwurf und Konzeption}\index{Entwurf und Konzeption}

\subsection{UML}\index{UML}

\subsection{Klassendiagramm}\index{Klassendiagramm}

\subsection{Use-Case Diagramme}\index{Use-Case Diagramme}


\section{Front-End Technologien}\index{Front-End Technologien}

Dieses Kapitel befasst sich mit Front-End-Technologien wie HTML, CSS, JavaScript und Angular, die für die Erstellung interaktiver, reaktionsfähiger und visuell ansprechender Benutzeroberflächen in unserer E-Commerce-Anwendung unerlässlich sind.

\subsection{HTML/CSS}\index{HTML/CSS}

HTML\footnote{HyperText Markup Language} and CSS\footnote{Cascading Style Sheets} sind die grundlegenden Technologien zur Erstellung und Gestaltung von Webseiten. HTML liefert die Struktur und den Inhalt einer Webseite, während CSS für die visuelle Darstellung, Layout, Farben und Schriftarten, verwendet wird. Sie ermöglichen es optisch ansprechende und gut strukturierte Webseiten zu erstellen. \cite{HTML/CSS:2024}

\subsection{JavaScript}\index{JavaScript}

JavaScript\footnote{https://www.javascript.com} ist eine vielseitige High-Level-Programmiersprache, die häufig für die Webentwicklung verwendet wird. Ursprünglich entwickelt, um Webseiten interaktiv zu gestalten, hat sie sich zu einer leistungsstarken Sprache entwickelt, die sowohl für die clientseitige als auch für die serverseitige Entwicklung verwendet werden kann. \cite{JavaScript:2024}

Die Programmiersprache ist auch für ihre dynamische Natur bekannt und ermöglicht es Entwicklern, interaktive Benutzeroberflächen und robuste serverseitige Anwendungen zu erstellen. Ihre Flexibilität und Ihre umfangreiches Ökosystem, zu dem Bibliotheken und Frameworks wie React, Angular und Node.js gehören, machen sie zu einem unverzichtbaren Werkzeug in der modernen Webentwicklung. 

\subsection{Angular}\index{Angular}

Angular\footnote{https://angular.io} ist ein Open-Source-Framework für Webanwendungen, das von Google entwickelt und gepflegt wird. Es wird für die Erstellung dynamischer, einseitiger Anwendungen (SPAs)\footnote{Single Page Application} mit TypeScript und HTML verwendet. Das Framework bietet eine robuste Plattform für die Entwicklung komplexer Anwendungen, indem es einen umfassenden Satz von Tools und Funktionen wie Datenbindung, Dependency Injection und eine modulare Architektur bietet. \cite{GOOGLE:2024}

Eine der Hauptstärken von Angular ist seine komponentenbasierte Struktur, die es Entwicklern ermöglicht, wiederverwendbare UI-Komponenten zu erstellen, die die Wartbarkeit und Skalierbarkeit verbessern. Darüber hinaus bietet Angular leistungsstarke Funktionen wie reaktive Programmierung mit RxJS\footnote{https://rxjs.dev}, Zustandsverwaltung und ein reichhaltiges Ökosystem von Bibliotheken, was es zu einer idealen Wahl für Anwendungen auf Unternehmensebene macht.

\subsubsection{Komponenten}

\subsubsection{Document Object Model (DOM)}

\section{Back-End Technologien}\index{Back-End Technologien}

In diesem Kapitel werden Back-End-Technologien untersucht, wobei der Schwerpunkt auf Spring Boot liegt, um robuste, skalierbare und sichere serverseitige Logik und APIs für unsere E-Commerce-Anwendung zu erstellen.

\subsection{Java}\index{Java}

\subsection{REST-API}\index{REST-API}

\subsection{Spring Boot}\index{Spring Boot}

\section{Datenbankstruktur}\index{Datenbankstruktur}

\section{Entwicklungsumgebung und Versionskontrolle}\index{Entwicklungsumgebung und Versionskontrolle}


