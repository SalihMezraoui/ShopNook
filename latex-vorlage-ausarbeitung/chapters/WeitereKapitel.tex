\chapter{Methodologie und Systementwurf }

Hier werden über folgende Punkte diskutiert:

\begin{itemize}
	\item Die Entwicklungskonzeption, die wir während der Implementation der Anwendung benutzt haben, sowie die Tools, Technologien und Methodologien. 
	\item Überblick über Spring Boot und seine wichtigste  Funktionalitäten, die auch in dem Projekt angewendet sein werden. 
	\item Detaillierte Erklärung der Struktur und Design der Anwendung (Bsp: Anlyse-/Entwurfsdiagram) 
	\item Datenbanken, UI-Design und unterschiedliche Komponenten des Systems. 
	\item Wie Spring Boot die Entwicklung und Implementation der verschiedenen Module vereinfacht. 
\end{itemize}


\section{Entwurf und Konzeption}\index{Entwurf und Konzeption}

In der Softwareentwicklung spielen Entwurf und Konzeption eine entscheidende Rolle. Diese Phase des Entwicklungsprozesses befasst sich mit der Strukturierung und Planung von Systemen, bevor die eigentliche Implementierung beginnt. 

Ziel ist es, ein klares und verständliches Modell zu erstellen, das die Anforderungen und Funktionen eines Systems abbildet. In diesem Kapitel werden UML und einige Diagrammtypen vorgestellt.

\subsection{UML}\index{UML}

Die Unified Modeling Language (UML) ist ein wesentliches Werkzeug im Bereich Entwurf. UML ist eine standardisierte Modellierungssprache, die eine Vielzahl von Diagrammen bietet, um unterschiedliche Aspekte eines Systems darzustellen. Diese Diagramme helfen dabei, komplexe Systeme zu visualisieren, zu dokumentieren und zu kommunizieren \cite{UML:2023}.

\subsection{Klassendiagramm}\index{Klassendiagramm}

\subsection{Use-Case Diagramme}\index{Use-Case Diagramme}


\section{Front-End Technologien}\index{Front-End Technologien}

Dieses Kapitel befasst sich mit Front-End-Technologien wie HTML, CSS, JavaScript und Angular, die für die Erstellung interaktiver, reaktionsfähiger und visuell ansprechender Benutzeroberflächen in unserer E-Commerce-Anwendung unerlässlich sind.

\subsection{HTML/CSS}\index{HTML/CSS}

HTML\footnote{HyperText Markup Language} and CSS\footnote{Cascading Style Sheets} sind die grundlegenden Technologien zur Erstellung und Gestaltung von Webseiten. HTML liefert die Struktur und den Inhalt einer Webseite, während CSS für die visuelle Darstellung, Layout, Farben und Schriftarten, verwendet wird. Sie ermöglichen es optisch ansprechende und gut strukturierte Webseiten zu erstellen \cite{HTML/CSS:2024}.

\subsection{JavaScript}\index{JavaScript}

JavaScript\footnote{https://www.javascript.com} ist eine vielseitige High-Level-Programmiersprache, die häufig für die Webentwicklung verwendet wird. Ursprünglich entwickelt, um Webseiten interaktiv zu gestalten, hat sie sich zu einer leistungsstarken Sprache entwickelt, die sowohl für die clientseitige als auch für die serverseitige Entwicklung verwendet werden kann \cite{JavaScript:2024}.\\
Die Programmiersprache ist auch für ihre dynamische Natur bekannt und ermöglicht es Entwicklern, interaktive Benutzeroberflächen und robuste serverseitige Anwendungen zu erstellen. Ihre Flexibilität und Ihre umfangreiches Ökosystem, zu dem Bibliotheken und Frameworks wie React, Angular und Node.js gehören, machen sie zu einem unverzichtbaren Werkzeug in der modernen Webentwicklung. 

\subsection{Angular}\index{Angular}

Angular\footnote{https://angular.io} ist ein Open-Source-Framework für Webanwendungen, das von Google entwickelt und gepflegt wird. Es wird für die Erstellung dynamischer, einseitiger Anwendungen (SPAs)\footnote{Single Page Application} mit TypeScript und HTML verwendet. Das Framework bietet eine robuste Plattform für die Entwicklung komplexer Anwendungen, indem es einen umfassenden Satz von Tools und Funktionen wie Datenbindung, Dependency Injection und eine modulare Architektur bietet \cite{GOOGLE:2024}.\\
Eine der Hauptstärken von Angular ist seine komponentenbasierte Struktur, die es Entwicklern ermöglicht, wiederverwendbare UI-Komponenten zu erstellen, die die Wartbarkeit und Skalierbarkeit verbessern. Darüber hinaus bietet Angular leistungsstarke Funktionen wie reaktive Programmierung mit RxJS\footnote{https://rxjs.dev}, Zustandsverwaltung und ein reichhaltiges Ökosystem von Bibliotheken, was es zu einer idealen Wahl für Anwendungen auf Unternehmensebene macht.

\subsubsection{Komponenten}


%\subsubsection{Document Object Model (DOM)}

\section{Back-End Technologien}\index{Back-End Technologien}

In diesem Kapitel werden Back-End-Technologien untersucht, wobei der Schwerpunkt auf Spring Boot liegt, um robuste, skalierbare und sichere serverseitige Logik und APIs für unsere E-Commerce-Anwendung zu erstellen.

\subsection{Java}\index{Java}

Java ist eine weit verbreitete Programmiersprache, die für ihre Plattformunabhängigkeit, Stabilität und umfassende Bibliotheken bekannt ist. Aufgrund ihrer Vielseitigkeit und Leistung wird Java häufig für die Entwicklung von Back-End-Anwendungen verwendet. In der E-Commerce-Anwendung (Shop-Nook) wird Java genutzt, um eine robuste und skalierbare serverseitige Logik zu gewährleisten.

\subsection{REST-API}\index{REST-API}

REST\footnote{Representational State Transfer} ist ein Architekturstil für die Entwicklung vernetzter Anwendungen. Er basiert auf einem zustandslosen, Client-Server- und cachefähigen Kommunikationsprotokoll, und in fast allen Fällen wird das HTTP-Protokoll verwendet. REST-APIs sind so konzipiert, dass sie einfach, leichtgewichtig und skalierbar sind, was sie zu einer beliebten Wahl für Webdienste macht.

\subsubsection{Einführung in REST}
REST-APIs bieten eine Möglichkeit, über HTTP auf die Funktionalität und Daten einer Anwendung zuzugreifen. Sie folgen den Prinzipien von REST, die auf Ressourcen und deren Repräsentationen basieren. Jede Ressource wird durch eine eindeutige URI (Uniform Resource Identifier) identifiziert und kann durch standardisierte HTTP-Methoden manipuliert werden:
\begin{itemize}
	\item GET: Abrufen von Ressourcen
	\item POST: Erstellen neuer Ressourcen
	\item PUT: Aktualisieren bestehender Ressourcen
	\item DELETE: Löschen von Ressourcen
\end{itemize}
REST-APIs sind leichtgewichtig und nutzen JSON (JavaScript Object Notation) oder XML (Extensible Markup Language) als Datenformat für den Datenaustausch.

\subsubsection{Vorteile von REST-APIs}

\begin{itemize}
	\item \textbf{Skalierbarkeit:} Durch das stateless Design können REST-APIs leicht skaliert werden. Jeder HTTP-Request enthält alle notwendigen Informationen, um ihn zu verarbeiten, ohne dass der Server den vorherigen Zustand kennen muss.
	\item \textbf{Flexibilität:} REST-APIs sind flexibel und können mit verschiedenen Datenformaten arbeiten. JSON ist das gebräuchlichste Format, da es leichtgewichtig und gut lesbar ist.
	\item \textbf{Interoperabilität:\footnote{Interoperabilität ist die Fähigkeit verschiedener Systeme oder Software, zusammenzuarbeiten und Informationen nahtlos auszutauschen\cite{wiki:listing}.}} REST-APIs nutzen standardisierte HTTP-Methoden, was ihre Interoperabilität mit verschiedenen Clients und Plattformen gewährleistet.
	\item \textbf{Leichte Integration:} REST-APIs lassen sich leicht in bestehende Systeme integrieren, da sie auf bekannten Webstandards basieren.
\end{itemize}

\subsubsection{Implementierung einer REST-API mit Spring Boot}
Spring Boot bietet umfassende Unterstützung für die Entwicklung von REST-APIs und vereinfacht deren Implementierung erheblich. Hier sind die wesentlichen Schritte zur Erstellung einer REST-API mit Spring Boot:
%\begin{enumerate}
%    \item \textbf{Projekt einrichten:} Spring Initializr wird verwendet, um ein neues Spring Boot-Projekt zu erstellen.
%    \item \textbf{Controller erstellen:} Ein Controller für den Checkout wird definiert, um die Endpunkte der API zu verwalten. Die Annotationen wie @RestController und @RequestMapping definieren die Routen.
%    \begin{lstlisting}[language=Java, caption={Controller-Implementierung für Checkout nach Bestellung in Java}\label{Controller.java}]
	%    @RestController
	%    @RequestMapping("/api/checkout")
	%    public class CheckoutController
	%    {
		%        private CheckoutService checkoutService;    
		%        public CheckoutController(CheckoutService checkoutService)
		%        {
			%            this.checkoutService = checkoutService;
			%        }
		%        @PostMapping("/purchase")
		%        public PurchaseResult submitOrder(@RequestBody Purchase purchase)
		%        {
			%            return checkoutService.submitOrder(purchase);
			%        }
		%    }
	%    \end{lstlisting}
%    \item \textbf{Modelle und Datenzugriff:} Erstellen Sie die Modelle, die die Datenstruktur Ihrer Anwendung repräsentieren. Verwenden Sie JPA-Repositories, um den Datenzugriff zu vereinfachen.
%\end{enumerate}




\subsection{Spring Boot}\index{Spring Boot}

Spring Boot ist ein Framework, das auf dem Spring Framework aufbaut und speziell entwickelt wurde, um schnelle, effiziente und skalierbare Anwendungen zu erstellen. Es bietet eine Vielzahl von Features, die den Entwicklungsprozess beschleunigen und vereinfachen, insbesondere für Java-basierte Webanwendungen und Microservices.

\subsubsection{Warum wurde Spring Boot ausgewählt ?}
Spring Boot wurde aufgrund der breiten Unterstützung in der Entwicklergemeinde ausgewählt. Hier sind einige der Hauptgründe:
\begin{itemize}
	\item \textbf{Produktivität:} Das Framework ermöglicht es auf das Schreiben von Geschäftslogik zu konzentrieren, anstatt sich um die Konfiguration von Technologien zu kümmern.
	\item \textbf{Microservices:} Es eignet sich hervorragend für die Entwicklung von Microservices-Architekturen, indem es Server wie Tomcat bietet, sodass die Anwendungen ohne externen Server laufen können.
	\item \textbf{Integration:} Es lässt sich nahtlos in andere Spring-Projekte und eine Vielzahl von Datenbanken integrieren.
\end{itemize}
Das Framework bietet auch viele Funktionalitäten, die es von anderen abheben, es hat viele weitere Vorteile, die hier genannt werden:

\begin{itemize}
	\item \textbf{Auto-Konfiguration:} Die Autokonfigurationsfunktion von Spring Boot konfiguriert Ihre Anwendung automatisch anhand der Abhängigkeiten, die Sie dem Projekt hinzugefügt haben. Dadurch wird die Notwendigkeit einer manuellen Konfiguration minimiert und die Entwicklung beschleunigt.
	\item \textbf{Eigenständige Anwendungen:} Spring Boot-Anwendungen können als eigenständige Java-Anwendungen verpackt werden, was die Bereitstellung einfacher und konsistenter macht.
	\item \textbf{Produktionstaugliche Funktionen:} Das Framework umfasst zahlreiche produktionsreife Funktionen wie Zustandsprüfungen, Metriken und externalisierte Konfiguration.	
\end{itemize}

\subsubsection{Wie funktionniert Spring Boot ?}

\begin{enumerate}
	\item \textbf{Initial Setup:} Mit Spring Initializr kann man schnell ein neues Spring Boot-Projekt mit allen erforderlichen Abhängigkeiten und Konfigurationen initialisieren.
	\item \textbf{Main Application Class:} Jede Spring Boot-Anwendung hat eine Hauptklasse, die mit @SpringBootApplication annotiert ist. Diese Annotation ist eine Kombination aus @Configuration, @EnableAutoConfiguration und @ComponentScan.
	\begin{lstlisting}[language=Java, caption={Main-Application-Class-Implementierung in Java}\label{mainClass.java}]
@SpringBootApplication
public class SpringBootEcommerceApplication {	
	public static void main(String[] args) {
		SpringApplication.run(SpringBootEcommerceApplication.class, args);
	}
}		
	\end{lstlisting}
	\item \textbf{Application Properties:} Spring Boot ermöglicht eine einfache Konfiguration durch application.properties-Datei. Dadurch wird die Konfiguration externalisiert, was die Verwaltung verschiedener Umgebungen erleichtert.
	\begin{lstlisting}[language=Java, caption={Implementierung von Application.properties Datei}\label{Application.properties}]
spring.datasource.driver-class-name=com.mysql.cj.jdbc.Driver
spring.datasource.url=jdbc:mysql://localhost:3306/mydb
spring.datasource.username=username
spring.datasource.password=password
spring.jpa.properties.hibernate.dialect=org.hibernate.dialect.MySQL8Dialect
spring.data.rest.base-path=/api
	\end{lstlisting}
\end{enumerate}




\section{Datenbankstruktur}\index{Datenbankstruktur}



\section{Entwicklungsumgebung und Versionskontrolle}\index{Entwicklungsumgebung und Versionskontrolle}


